\chapter{Analyse}
\label{ch:analysis}

Afin de structurer et formaliser les besoins des utilisateurs dans le but d'assurer la compréhension entre les différents partis, j'ai choisi de les représenter sous forme de diagrammes \gls{a-uml}. L'\gls{a-uml} est un standard d'industrie qui permet de représenter des informations de manière univoque. Bien que ce soit sensé être un standard, les différentes implémentations disponibles sur le marché se distinguent les unes des autres et il peut être utile de noter que j'ai utilisé le logiciel \guillemotleft{} Visual Paradigm Modeler Edition \guillemotright{} en version 15.2.

\section{Les cas d'utilisation}
\label{sec:use-cases}

\paragraph{}
Le seul besoin métier est la validation de fichiers Excel contenant des données utiles à l'accomplissement du travail de l'utilisateur. Toutefois, la mise en place d'un site web implique la gestion d'autres aspects comme la sécurité ou l'administration du site web.

\paragraph{}
On distingue trois types d'utilisateurs qui n'ont pas accès aux mêmes fonctionnalités:
\begin{itemize}
    \item Le \textbf{visiteur}: Il n'a accès à rien si ce n'est l'accès à l'authentification afin de gagner les privilèges de l'\textbf{utilisateur}.
    \item L'\textbf{utilisateur}: Il a accès aux fonctionnalités qui font le coeur de l'application. Cette dernière est conçue pour lui.
    \item L'\textbf{administrateur}: Il a tous les droits de l'utilisateur normal avec en plus la possibilité d'accéder aux outils d'administration qui permettent notamment de modifier les paramètres du serveur ainsi que les utilisateurs eux-mêmes.
\end{itemize}

À partir d'ici, je mettrai ces trois termes en gras lorsque je ferai explicitement référence aux définitions présentes ci-dessus plutôt qu'à celles de la langue française.

\subsection{La liste des cas d'utilisation}
\label{subsec:use-cases-list}

\begin{figure}[ht]
    \centering
    \includegraphics[width=0.8\textwidth]{images/diagrams/use-cases-macro.png}
    \caption{Les cas d'utilisation}
    \label{fig:use-cases-macro}
\end{figure}

\paragraph{}
Le diagramme \ref{fig:use-cases-macro} liste les différents cas d'utilisation et en l'analysant avec attention, on peut observer certains faits qui sont à première vue contrintuitifs.

\paragraph{}
Dans de nombreuses applications, on pourrait considérer une sorte d'héritage entre les différents niveaux de privilèges parmi les utilisateurs. Or, ce n'est pas le cas ici. En effet, seul le \textbf{visiteur} a accès aux fonctionnalités pour s'enregistrer et se connecter. Plus surprenant, l'\textbf{administrateur} n'a pas accès aux fonctionnalités du coeur de l'application que sont la validation et le traitement des classeurs Excel. Encore plus étonnant, l'administrateur ne peut ni accéder à son profil ni changer son mot de passe.

\begin{figure}[ht]
    \centering
    \includegraphics[width=0.8\textwidth]{images/screenshot/screenshot-user-admin-page.png}
    \caption{Une capture d'écran de la page de gestion des utilisateurs}
    \label{fig:user-admin-page}
\end{figure}

\paragraph{}
Cela est dû au fait que les permissions sont gérées par un système de licences. Une personne peut cumuler les licences. Ainsi, plutôt que de faire hériter un rôle d'un autre, nous avons choisi de lier les permissions aux licences et de donner autant de licences que nécessaire aux personnes. Cela permet une gestion plus fine des autorisations et aussi plus simple à mettre en place.

Grâce à cela, la nouvelle application peut être déployée avec la base des utilisateurs existante complète sans risquer de compromettre les ressources qu'elle expose. En effet, seules les personnes qui recevront la licence appropriée auront réellement accès à l'application.

De ce fait, un \textbf{visiteur} n'est pas une personne sans licence, mais une personne qui n'a aucune licence adaptée.

Dans les faits, un \textbf{administrateur} aura toujours la licence d'un \textbf{utilisateur}. C'est ce que l'on peut observer sur l'image \ref{fig:user-admin-page}. Il a donc la double casquette d'\textbf{administrateur} et d'\textbf{utilisateur}.

\paragraph{}
Le traitement inclut la validation. Bien que ces deux fonctionnalités soient fondamentalement distinctes, nous avons fait le choix d'inclure la validation comme première étape du traitement. Faire ainsi permet d'assurer la validité des données traitées et exclut toute erreur humaine.

\paragraph{}
Une chose qu'il n'est pas possible de voir sur ce diagramme \ref{fig:use-cases-macro} est que la validation et le traitement des classeurs sont des cas d'utilisation abstraits. En effet, il est nécessaire de les spécialiser sans quoi ils ne représentent rien.

Il faut se pencher sur des schémas plus détaillés pour comprendre ce que ces cas d'utilisation couvrent. C'est le sujet de la sous-section \ref{subsec:spreadsheet-use-case}.


\subsection{L'authentification}
\label{subsec:auth-feature}

\paragraph{}
L'authentification mise en place doit être compatible avec le système d'authentification présent sur les services d'Altissia. C'est particulièrement facile, car c'est une authentification dite sans serveur. Cela veut dire qu'un client peut prouver son identité sans qu'un serveur tiers confirme les droits auxquels le \gls{g-client} prétend. 

\begin{figure}[h]
    \centering
    \includegraphics[width=0.7\textwidth]{images/diagrams/gw-archi.png}
    \caption{L'organisation des services et leurs communications (non \gls{a-uml})}
    \label{fig:gw-archi}
\end{figure}

\paragraph{}
Pour comprendre comment c'est possible, il faut d'abord s'intéresser à la manière dont les différents services communiquent entre eux. Comme on peut le voir sur le diagramme \ref{fig:gw-archi}, tous les \glspl{g-server} sont cachés derrière un unique point d'entrée que l'on appelle la passerelle. La passerelle vérifie que les demandes faites aux \glspl{g-server} sont authentifiées. Les demandes faites entre \glspl{g-server} ne sont pas vérifiées, car les \glspl{g-server} se font mutuellement confiance.

\begin{figure}[h]
    \centering
    \includegraphics[width=0.7\textwidth]{images/screenshot/jwt-good-secret.png}
    \caption{Un \gls{a-jwt} à gauche et les données décodées à droite}
    \label{fig:jwt-good}
\end{figure}
\begin{figure}[h]
    \centering
    \includegraphics[width=0.7\textwidth]{images/screenshot/jwt-bad-secret.png}
    \caption{Un \gls{a-jwt} avec les mêmes données que sur la figure, \ref{fig:jwt-good} mais pas la même signature}
    \label{fig:jwt-bad}
\end{figure}

\paragraph{}
Un \gls{g-client} qui veut s'authentifier fait sa demande à la passerelle.
La passerelle consulte le service des utilisateurs et si l'identifiant et le mot de passe concordent, elle approuve l'authentification.
Elle envoie alors un \gls{a-jwt}, dont on peut observer un exemplaire sur la figure \ref{fig:jwt-good} qui est une sorte de passeport qui déclare l'identité et les droits de l'utilisateur.
Ce \gls{a-jwt} est lisible par tous et est infalsifiable. Son contenu est accompagné d'une signature qui est le résultat de ce même contenu passé par une \gls{g-hash-func}.
La seule manière de reproduire cette signature est de disposer de la \gls{g-hash-func} et du secret.
La figure \ref{fig:jwt-bad} illustre ce que l'on obtient sans le bon secret.
Cette formule est présente sur le \gls{a-jwt} et est donc accessible à tous.
Le secret lui n'est connu que par la passerelle et il n'y a donc qu'elle qui peut produire cette signature.

\paragraph{}
À chaque requête, un client accompagne sa demande de son \gls{a-jwt} et la passerelle reproduit la signature.
Si la signature produite est la même que celle inscrite sur le \gls{a-jwt}, la passerelle transmet la demande, sinon elle la rejette.
Et lorsque les \glspl{g-server} doivent communiquer entre eux pour remplir la demande du client, ils font confiance aux droits indiqués, car ils ont été vérifiés par la passerelle.

\paragraph{}
Pour que cela fonctionne, la nouvelle application cliente doit:
\begin{enumerate}
    \item Adresser toutes ses demandes à la passerelle
    \item Implémenter la demande d'authenfication
    \item Fournir le \gls{a-jwt} à chaque requête
\end{enumerate}
Tandis que le nouveau \gls{g-server} doit juste contrôler que les droits d'un utilisateur correspondent au niveau d'autorisation requise pour les ressources qu'il demande.


\subsection{L'internalisation}
\label{subsec:i18n}

\paragraph{}
Chaque valeur textuelle statique\fnmark{} doit être traduite. Seuls des employés d'Altissia auront accès à cette application, il n'est donc pas nécessaire de traduire l'application dans toutes les langues supportées par les autres applications d'Altissia\fnmark{}.
\fntext{Toutes données qui ne viennent pas d'une base de données sont dites statiques, car elles ne peuvent pas changer pendant la vie de l'application.}
\fntext{Ce qui aurait fait une trentaine de langues dont je n'en connais que trois...}

Il est par contre obligatoire d'être compatible avec le système de localisation qui est déjà mis en place.

Altissia embarque les mécanismes nécessaires à la localisation dans chaque application. Une application peut donc changer de langue sans contacter de \gls{g-server}.

De plus, les informations de localisations sont stockées sur l'application web PhraseApp.
Les localisations devront donc être téléchargées depuis cette dernière.
Le format privilégié pour les applications clientes est le format \gls{a-json}, que l'application devra donc utiliser.

\paragraph{}
L'application cliente devra donc implémenter la localisation seule. C'est le sujet qui est abordé dans la sous-section \ref{subsec:i18n-imp}


\subsection{La validation et le traitement des classeurs}
\label{subsec:spreadsheet-use-case}

\paragraph{}
Le coeur de l'application est la validation et le traitement des classeurs.
La validation vérifie que les données respectent les règles auxquelles elles doivent souscrire pour permettre l'usage auquel elles sont destinées.
Le traitement utilise les données et changent l'état des bases de données des services qui vont consommer ces données.

\paragraph{}
Le travail que je présente ici est un \gls{g-framework}, une sorte de boite à outils conçues pour répondre à une gamme de besoins.
Il n'a donc pas d'utilité propre; on peut construire un banc avec une boite à outils et des planches mais on ne peut pas s'asseoir que sur le banc.
Les besoins auxquelles doivent répondre ces outils, définis dans la sous-section \ref{subsec:business-needs}, sont des exemples venant d'applications propriétaires dont je ne peux pas discuter les spécifications exactes ici.
J'ai donc imaginé un exemple fictif qui a besoin des mêmes outils pour être satisfait.
Cet exemple aura aussi l'avantage de répondre à une logique métier beaucoup plus simple que les cas réels.
Pour rappel, vous pouvez consulter un exemple de cas réels dans l'annexe \ref{ch:leveltest-sample}.

\subsubsection{La validation d'un classeur}
\label{subsubsec:spreadsheet-validation-case}

\paragraph{}
Le classeur imaginé a deux feuilles, une qui contient des informations sur des apprenants et une sur des services.
Les apprenants sont des personnes pour lesquelles l'\textbf{utilisateur} veut créer des comptes pour qu'ils puissent suivre des cours sur nos plateformes.
Les services décrivent les accès que l'\textbf{utilisateur} veut donner à ces apprenants.

\begin{figure}[ht]
    \centering
    \includegraphics[width=0.7\textwidth]{images/screenshot/sheet-users.png}
    \caption{Un modèle valide de la feuille "registrants" qui liste les futurs apprenants}
    \label{fig:sheet-learners}
\end{figure}

\paragraph{}
Sur la figure \ref{fig:sheet-learners}, on peut voir les quatre attributs du futur élève:
\begin{itemize}
    \item First Name: Le prénom de l'apprenant est optionnel et s'il est présent, il doit être composé de lettres \gls{g-unicode} dont la première doit être une majuscule si il existe une version majuscule de cette lettre\fnmark{}.
    \fntext{En réalité, il serait contraignant de limiter les noms à ce format tant la diversité est grande dans le monde. Ce cas n'est qu'un exemple d'utilisation d'une \gls{g-regex}}
    \item Last Name: Le nom de l'apprenant est obligatoire et respecte les mêmes règles que le prénom.
    \item Email: L'email est obligatoire, doit être un email valide et doit être unique dans la feuille.
    \item Interface language: La langue est obligatoire et doit faire parti d'une énumération définie par le code de l'application.
\end{itemize}

\paragraph{}
Voici un extrait pertinent du code qui définit les langues, dont vous pouvez lire l'entièreté dans l'annexe \ref{ch:language-java}:

\lstinputlisting[language=java, firstnumber=13, firstline=13, lastline=19]{pages/appendix/Language.java}

On y observe qu'il y a deux sortes de langues, les langues d'apprentissages et les langues d'études.
Une langue d'étude est implicitement aussi une langue d'interface.
Il est impératif de se servir de cette classe et non d'une copie car il faut suivre ses évolutions.

\begin{figure}[ht]
    \centering
    \includegraphics[width=0.7\textwidth]{images/screenshot/sheet-services.png}
    \caption{Un modèle valide de la feuille "services" qui liste les licences nécessaires aux apprenants}
    \label{fig:sheet-services}
\end{figure}

\paragraph{}
La feuille décrivant les services, dont un exemple est illustré par la figure \ref{fig:sheet-services}, contient les champs suivants:
\begin{itemize}
    \item Email: L'email est obligatoire et doit avoir été défini par un utilisateur sur l'autre feuille pour pouvoir être utilisé.
    \item License: La licence est obligatoire et doit être un membre de l'énumération \textit{Service} définie dans le code.
    \item Language: La langue est obligatoire, doit être une langue d'étude et donc faire parti d'un ensemble restreint de l'énumération \textit{Language}.
    \item Duration: Cette durée, exprimée en jours entiers, doit être comprises entre 0 et 365.
\end{itemize}
La classe \textit{Service} n'a rien de particulier et elle définit les éléments suivants: \textit{COURSE}, \textit{ASSESSMENT}, \textit{NEWS}, \textit{VISIO}.

La combinaison de l'email, la licence et la langue doivent être unique dans la feuille.

\paragraph{}
Outre les règles au sein des feuilles, des règles sont appliquées sur l'ensemble du classeur:
\begin{itemize}
    \item Un email défini dans la feuille "registrants" doit être utilisé dans la feuille "services"
    \item Un email utilisé par la feuille des "services" doit être défini dans la feuille "registrants"
    \item Certaines erreurs peuvent être reclassées comme des avertissements, qui peuvent être ignorées pour procéder au traitement
\end{itemize}

\paragraph{}
Enfin, certaines règles sont externes au contenu du seul classeur:
Un email défini dans la feuille "registrants" ne peut pas être déjà présent dans la base des données des apprenants; l'apprenant doit être nouveau.

\subsubsection{Le traitement d'un classeur}
\label{subsubsec:spreadsheet-processing-case}
Le traitement de ce classeur consiste à créer les apprenants, leurs licences et les liens entre les deux à partir des informations contenues dans le classeur.


