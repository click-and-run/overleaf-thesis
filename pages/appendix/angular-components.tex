\chapter{Les composants Angular pour la validation et le traitement des fichiers}
\label{ch:angular-components}

\Gls{g-angular} prévoit différent types d'objets en fonction de leur usage:
\begin{itemize}
    \item Le composant: un élément graphique avec un comportement associé
    \item La directive: un comportement que l'on peut associé à un composant
    \item Le service: un service disponible auprès de tous les autres objets
    \item Le module: il regroupe d'autres objets afin de former un ensemble cohérent
\end{itemize}

Pour la validation et le traitement des classeurs, j'ai implémenté deux modules, un module pour la gestion des fichiers et un pour la gestion des classeurs.

\section{La gestion des fichiers}
\label{sec:file-management}

\paragraph{}
Il existe de nombreuses librairie pour la gestion des fichiers sur internet mais toutes me posaient soit le problème de la compatibilité ou un problème de qualité.
J'ai donc fait le choix d'en concevoir une moi-même.

\paragraph{}
Une directive se charge du gros du travail et implémente tout les comportements nécessaires.
Tandis qu'un composant gère l'aspect visuel et expose les fonctionnalités de la directive qu'il utilise.

La directive:
\lstinputlisting[]{pages/appendix/code/dragndrop.directive.ts}

Le code gérant la logique du composant:
\lstinputlisting[]{pages/appendix/code/dragndrop.component.ts}

Le code gérant la structure visuelle du composant:
\lstinputlisting[language=html]{pages/appendix/code/dragndrop.component.html}
Bien que mon cas d'utilisation ne requière, pour le moment, qu'un seul fichier à la fois, le composant est déjà prêt pour en gérer plusieurs.
