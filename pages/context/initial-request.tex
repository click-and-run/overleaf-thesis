\paragraph{}
Des tâches sont actuellement effectuées par une bibliothèque de scripts \href{https://ant.apache.org/}{Apache Ant}\footnote{Apache Ant est un outil permettant de construire une application Java à partir de son code source. Une alternative plus connue est \href{https://maven.apache.org/}{Apache Maven}.}.
Certaines sont effectuées avec des scripts codés en \href{https://www.python.org/}{Python}\footnote{Langage de programmation très populaire pour créer des scripts.}.
Enfin d'autres sont réalisées avec le socle d'application\footnote{Socle d'application ou \textit{Framework} en anglais; ensemble cohérent de composants logiciels permettant de construire les fondations d'un logiciel\cite{wikipedia_framework_2019}.} Spring\footnote{Spring est un socle d'application Java.} Batch\footnote{Spring Batch est un composant du \textit{Framework} Spring permettant un traitement par lots de grande quantité de données.}.

\paragraph{}
Malheureusement, seuls des développeurs sont capables d'utiliser ces outils.
En effet, ils ne disposent pas d'interface graphique ou de connexion avec des programmes classiques\footnote{Il est existe une connexion entre certains modules Apache Ant et Excel mais c'est loin d'être ergonomique et est finalement peu utilisé.}.
De plus, rien ne centralise ces applications, ce qui ne facilite pas non plus la vie des programmeurs.
Ces programmes ont été écris au moment où ils étaient nécessaire sans trop de pensées vers l'architecture globale du système applicatif futur.
Ce qui rends ces applications assez chers à maintenir et peu plaisantes à faire évoluer.

\paragraph{}
Le but principal du nouveau logiciel est de centraliser tous ces processus, fournir un cadre les rendant plus facile à développer et à faire évoluer et permettant leur utilisation au travers d'une interface graphique par un acteur métier.
