\subsection{Les outils compatibles}
\label{subsec:compatible-tools}

\paragraph{}
Pour valider une ligne d'un fichier Excel, j'ai choisi de procéder en deux étapes:
\begin{itemize}
    \item Transférer les données de la ligne sur un modèle d'objet
    \item Valider l'objet
\end{itemize}
C'est une stratégie qui est très usitée pour la validation de données avant de les stocker dans une base de données ou encore lors de réception de données d'un utilisateur.
L'objet est une abstraction qui permet de représenter de manière intuitive les données sans tenir compte de l'implémentation sous-jacente du stockage utilisé.

\paragraph{}
J'ai choisi d'utiliser un format établi qui est utilisé par de nombreux outils existants sur le marché: \textit{javax.validation.constraints}.
Ce choix permettra instantanément de disposer de centaines de règles déjà écrites et utilisées dans l'industrie.

Dans l'extrait de code suivant, on peut observer à quoi ressemble un modèle et observer quelques annotations.
\lstinputlisting[language=java, firstnumber=14, firstline=14, lastline=37]{pages/appendix/code/RegistrantRow.java}
Les annotations \lstinline{@NotNull} et \lstinline{@Pattern} sont directement définies par la librairie \textit{javax.validation.constraints}.
Les annotations \lstinline{@Email} et \lstinline{@Length} viennent de l'\gls{a-orm} \Gls{g-hibernate}.
L'annotation \lstinline{@InterfaceLanguage} a été créée par mes soins.
Les autres annotations ne servent pas à valider les données, mais à faire le lien entre la ligne dans le fichier Excel et le modèle d'objet.
