\subsection{Les interviews}
\label{subsec:interviews}

\paragraph{}
J'ai eu une première interview avec Grégory pour définir le besoin général et le contexte qui l'entoure.
Il explique qu'en facilitant la vérification et la conversion des nouveaux contenus des cours, l'application permettrait d'accélérer l'intégration des nouveaux contenus de cours dans nos plateformes.

\paragraph{}
Ensuite, j'ai eu une réunion avec Renaud et Sophie.
A deux, ils ont défini précisément la problématique des outils existants et pourquoi on veut les remplacer.

Renaud connaît très bien l'outil que l'on veut remplacer, Altissia launcher, il m'a expliqué les difficultés rencontrées pendant son développement et comment il a terminé dans son état actuel.

Sophie a établi la liste des manquements du logiciel actuel qu'il faudrait suppléer et les fonctionnalités qu'elle voudrait ajouter.

\paragraph{}
Par la suite, nous avons continué de nous rencontrer dans le cadre des réunions SCRUM:
\begin{itemize}
    \item Le \textit{daily standup}: Courte réunion quotidienne pour informer de l'avancement et toute difficulté rencontrée
    \item Le \textit{sprint planning}: Réunion bimensuelle où l'on planifie les deux prochaines semaines
    \item Le \textit{backlog refinment}: Réunion bimensuelle où l'on clarifie la définition des fonctionnalités pour s'assurer qu'elles sont correctement comprises et prêtes à être implémentées.
    \item La \textit{Review}: Réunion bimensuelle en fin de sprint où l'on présente les résultats produits durant la période à toutes personnes intéressées et prends note des commentaires.
\end{itemize}
