De nombreuses tâches relativement manuelles doivent être effectuées pendant les cycles de dévelopement, publication,
mises en production et autres.
Ces tâches ne sont pas forcémment compliquées mais elles sont obligatoires et parfois critiques.
Elles doivent être accomplies avec attention et rigueur.

Ces tâches sont par exemple la validation du contenu d'un fichier excel, l'extraction de ce même contenu vers une base des données,
la création de licenses ou encore l'analyse statistique de la base de données.

Il est impératif de pouvoir s'assurer de leur bon déroulement et de pouvoir obtenir un rapport détaillé
de leur exécution.

\section{Le client}
\label{sec:customer}

    \paragraph{}
    Altissia est une entreprise travaillant dans le domaine de l'apprentissage des langues étrangères.
    Elle propose des formations en lignes.
    Elle développe les plateformes informatiques, crée le contenu de cours, fournit du support aux utilisateurs et héberge une
    partie de son infrastructure.
    Son catalogue contient 31 langues d'interfaces et 26 langues d'apprentissages\footnote{Information vraie au 11 Janvier 2019.}.
    Elle se distingue de ses concurrents par son identité académique en proposant des cours complets permettant de passer de débutant à expert.

    \paragraph{}
    Son siège principal se trouve en Belgique mais elle a aussi des sièges en France, au Brésil, au Canada et au Maroc.
    Environ 80 employés travaille en son sein dont 60 sur le site de Louvain-la-Neuve.
    Nous collaborons avec le CLL et l'UCL dans la création de notre contenu.

    Nos clients sont principalement des institutions comme l'Union Européenne ou la Région Wallonne, des écoles ou encore
    des entreprises privées qui veulent former, respectivement, leurs citoyens, leurs étudiants et leurs collaborateurs.

    \paragraph{}
    Altissia propose des cours sur sites internets et applications mobiles, des classes à distance, des articles de presses
    avec traduction assistée, un réseau social et des tables de conversations.
    Et fait donc face à de nombreux concurrents présents sur différentes plateformes.
    On peut citer Rosetta Stone, Duolinguo, Babel, MemRise, italki ou encore HelloTalk mais la liste est sans fin.

    \paragraph{}
    L'application sera utilisée par le service linguistique et informatique qui sont conjointement responsables du contenu de cours
    et du développement des plateformes de cours.
    Ces deux services comptabilisent une quarantaine de personnes et comporte des profils de linguistes, développeurs informatiques,
    concepteurs d'interfaces, administrateurs systèmes, chefs de projets et analyste commerciaux.

\section{La demande initiale}
\label{sec:initial-demande}

    \paragraph{}
    Certaines tâches sont actuellement effectuées par un projet de scripts \href{https://ant.apache.org/}{Apache Ant}.
    Malheureusement, seuls des développeurs sont capables d'utiliser cet outil avec la ligne de commande ou leur IDE.
    De plus ces scripts ne sont pas facilement maintenables et ne couvrent plus tous les besoins.

    \paragraph{}
    Dans un premier temps, la nouvelle application devra reprendre une fonctionalité du projet existant.
    La première fonctionalité choisie sera la validation du contenu d'un fichier Excel;
    il faut tester si il adhère à un ensemble de règles.
    De plus, l'utilisation de cette fonctionalité doit être entièrement possible par une personne non développeur.
    Enfin, le rapport d'exécution de chaque tâche doit être sauvegardé.

\section{Les produits existants}
\label{sec:existing-products}

    \paragraph{}
    Différents produits auraient pu nous intéresser comme par exemple
    \href{https://ifttt.com/}{IFTTT},
    \href{https://zapier.com/}{Zapier} ou
    \href{https://flow.microsoft.com/fr-fr/}{Microsoft Flow}.
    Ces outils gère tous l'aspect de journalisation, leurs interfaces sont conviviales et elles remplissent certains de nos besoins métiers.
    Quoiqu'il arrive, certains de nos besoins métiers sont uniques et nous devrons les implémenter nous même à travers des modules.
    On pourrait par exemple tirer parti des \href{https://powerapps.microsoft.com/fr-fr/}{PowerApps} de \href{https://flow.microsoft.com/fr-fr/}{Microsoft Flow}.

    \paragraph{}
    Toutefois, dès lors que l'on sort de l'écosystème de l'outil que nous aurions choisis, le framework de ce dernier ne nous apporterait que peu d'aide.
    Beaucoup de nos besoins métiers ne correspondent pas exactement à ces outils et nous serions laissés à nous mêmes.
    Aussi, un problème majeur est le système d'authentification de nos différentes ressources internes;
    Un accès \href{https://fr.wikipedia.org/wiki/Samba_(informatique)}{SAMBA} ou
    une authentification sur un \href{https://fr.wikipedia.org/wiki/Annuaire}{annuaire} ne poserait pas de soucis
    mais l'intégration avec nos micro services serait très complexe à mettre en oeuvre.

    \paragraph{}
    Ainsi, en développant un outil qui corresponde exactement à nos besoins et sur lequel nous avons totalement la main, nous espérons obtenir un meilleur résultat.
