L'élaboration de ce cahier des charges s'est faite en deux phases. 
Dans un premier temps, j'ai participé à des réunions avec Grégory, Renaud et Sophie afin de démarrer le projet. 
Et dans un second temps, nous avons ajouté et adapté des fonctionnalités au fur et à mesure que le projet a avancé.

\paragraph{}
Altissia travaille selon les principes agiles\fnmark et plus précisément la méthodologie SCRUM\fnmark.
Ces principes sont nés d'un constat sur les projets informatiques, la majorité échoue\cite{standish_standish_nodate} et la plupart du temps c'est parce que les besoins sont mal définis ou parce que la communication entre les parties prenantes est mauvaise (voir figure \ref{fig:why-projects-fails}).

\begin{figure}[ht]
    \centering
    \includegraphics[scale=.8]{images/why-projects-fail.png}
    \caption{Role of Requirements in Software Project Failures. Source: ESI International Survey of 2000 Business Professionals, 2005.}
    \label{fig:why-projects-fails}
\end{figure}

\fntext{Dans le contexte du développement logiciel, l'agilité définit un cadre de travail\cite{noauthor_methode_2018}.
Un point central est le développement par itération dont chacune produit un résultat critiquable par le client.
Ceci a pour but de limiter les efforts potentiellement perdus si le résultat ne correspond pas aux attentes.}
\fntext{Une méthode Agile spécifique qui a la particularité d'organiser le travail sous forme d'échéances à court terme que l'on appelle les sprints\cite{noauthor_guide_2013}.}

\paragraph{}
Pour répondre à ces enjeux, le service informatique d'Altissia a adopté la méthode SCRUM.
Cela consiste à travailler de manière itérative, de régulièrement produire des résultats intermédiaires utilisables, d'impliquer les différents acteurs, d'obtenir leurs commentaires et d'adapter la direction du projet selon ceux-ci.

\paragraph{}
C'est donc pour cela que le cahier des charges n'est pas complet et définitif au début du projet, mais qu'il s'étoffe au fur et à mesure.
