\paragraph{}
Altissia développe des sites web pour apprendre les langues en ligne.
Ces sites web utilisent de très nombreuses sortes de données:
\begin{enumerate}
    \item Questions de test de niveau
    \item Leçons
    \item Activités
    \item Vidéos
    \item Images
    \item Sons
    \item Articles de presse
    \item Quiz
    \item Dictées
    \item Autres
\end{enumerate}
Ces données sont loin d'être figées, elles évoluent pour s'adapter aux intérêts des apprenants, aux évolutions des langues, pour apporter des corrections ou plus généralement pour améliorer l'expérience de l'utilisateur.

Dans un monde idéal, toutes ces données seraient manipulées par les acteurs métiers au travers d'outils dédiés qui modifieraient directement les données dans leur contexte d'exploitation.
Toutefois, afin de satisfaire les exigences et désidératas de nos clients dans les temps impartis, la conception du contenu supplante le développement de son outil d'édition.

\paragraph{}
Toutes ces données prennent presque toujours un format différent entre le moment où elles sont travaillées et le moment où elles sont disponibles sur une plateforme.
Par exemple, une question d'un test de niveau est représentée par une ligne d'un fichier Excel pour les linguistes qui les éditent, mais correspond à plusieurs entrées dans une base de données Elasticsearch pour l'application.

\paragraph{}
Il est donc impératif d'effectuer la conversion entre ces différents formats afin de nourrir les applications.
