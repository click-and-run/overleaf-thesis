\subsection{Des outils prêts à l'emploi}
\label{subsec:ready-tools}

\paragraph{}
La plupart des règles de validation ne sont pas propres à un seul classeur, il faut donc les concevoir de façon à ce qu'elles soient utilisables dans d'autres classeurs.

\paragraph{}
Je ne vais donc pas implémenter une règle pour vérifier un format de chaine de caractère précis, mais une règle pour vérifier les formats de manière générale.

Je n'ai pas créé une règle qui vérifie qu'un nombre est compris entre 0 et 365, mais deux règles: une pour vérifier qu'une valeur est plus grande qu'une borne et une qui vérifie que la même valeur est inférieure à une autre borne.

Vérifier qu'une valeur est un service valide revient à vérifier qu'une valeur est un élément d'un ensemble.

Vérifier que le nom est long de deux à cinquante caractères résulte en une règle qui vérifie la taille d'une chaine de caractères.

Vérifier que l'email est bien défini revient à vérifier qu'il n'est pas \gls{g-null}.

Et j'ai répété de principe pour toutes règles.

