\paragraph{}
Ces quatre dernières années, j'ai suivi les études pour acquérir le bachelier en informatique de gestion.
Le développement informatique me donne l'impression que tout peut être automatisé.
Ce présent travail est un argument en faveur de cette assertion et un formidable défi aux compétences et connaissances acquises lors de mon parcours académique.

\paragraph{}
Je travaille à Altissia depuis deux ans.
Je développe des nouvelles fonctionnalités, corrige des bogues et réalise différentes tâches opérationnelles.
C'est cette dernière catégorie qui nous intéresse ici.

Elle contient des tâches récurrentes et courtes, réalisables en deux jours ou moins.
Elles constituent une étape indispensable au déploiement des nouveaux contenus sur les applications webs.
Mais si on additionne le nombre de fois qu'il faut les effectuer, elles sont finalement chronophages.

\paragraph{}
De plus la personne qui a besoin du résultat de cette tâche est souvent différente de celle qui est capable de l'effectuer, rendant la première dépendante de la deuxième.

\paragraph{}
Enfin, ces tâches utilisent des technologies et ressources variés et sont dispersées dans de multiples composants logiciels de l'entreprise.

\paragraph{}
Je pense que centraliser ces tâches et normaliser leur utilisation au travers d'une application unique permettrait d'augmenter la qualité des tâches opérationnelles, simplifierait leur mise en oeuvre et réduirait la friction dans les processus de déploiement.
