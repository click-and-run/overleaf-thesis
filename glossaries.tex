\newglossaryentry{g-null}{
    name=null,
    description={est la valeur que l'on attribue à une variable pour dénoter l'absence de valeur, ce qui est différent d'une valeur de zéro pour un nombre par exemple}
}
\newglossaryentry{g-edi}{
    name=environnement de développement intégré,
    description={est un logiciel qui intègre les outils pour écrire le code source ainsi que les outils pour le transformer en application prête pour l'utilisateur final}
}
\newglossaryentry{g-server}{
    name=serveur,
    description={est une entité qui fournit une ressource à une entité qui lui a demandé cette ressource}
}
\newglossaryentry{g-client}{
    name=client,
    description={est une entité qui reçoit une ressource d'une entité à laquelle elle lui a demandée cette ressource}
}
\newglossaryentry{g-jwt}{
    name=json web token,
    description={est un standard industriel permettant de représenter des prétentions entre deux partis de manière sécurisée}
}
\newglossaryentry{g-hash}{
    name=hash,
    description={est le résultat d'une fonction de hachage}
}
\newglossaryentry{g-hash-func}{
    name={fonction de hachage},
    description={est une fonction qui calcule une empreinte numérique à partir d'une entrée donnée. Le résultat de cette fonction, le hash, sera toujours le même pour la même entrée et jamais le même pour une entrée différente. Il est très difficile, voir impossible de trouver l'entrée sur base du résultat, la fonction est donc à sens unique. Afin de pouvoir standardiser de telles fonctions dans le contexte de la sécurité informatique, elles sont conçues pour utiliser un secret qui modifie le résultat de la fonction. De telle sorte qu'il n'est pas possible de reproduire le comportement d'une fonction son connaître sans secret, cela alors que la fonction elle-même est connue}
}
\newglossaryentry{g-json}{
    name={\gls{g-javascript} Object Notation},
    description={est un format textuel utilisée par le langage de programmation \gls{g-javascript} pour représenter des données structurées}
}
\newglossaryentry{g-framework}{
    name={socle d'applications},
    description={est un ensemble cohérent de composants logiciels destinés à créer les fondations d'un logiciel\cite{wikipedia_framework_2019}}
}
\newglossaryentry{g-unicode}{
    name=Unicode,
    description={est un standard informatique qui permet de représenter les caractères utilisés de presque toutes les langues de la Terre}
}
\newglossaryentry{g-regex}{
    name=expression régulière,
    description={est une chaine de caractères, qui décrit, selon une syntaxe précise, un ensemble de chaines de caractères possibles et définit donc un format de chaine de caractères\cite{noauthor_expression_2019}}
}
\newglossaryentry{g-bool-func}{
    name=fonction booléenne,
    description={est une fonction qui détermine une réponse vraie ou fausse selon les paramètres qui lui sont donnés}
}
\newglossaryentry{g-agile}{
    name=Agile,
    description={est un cadre de travail pour la conception de logiciels. Il encourage notamment un travail itératif et une communication constante avec le client\cite{noauthor_methode_2018}}
}
\newglossaryentry{g-scrum}{
    name=scrum,
    description={est le mot anglais pour mêlée et dans le contexte de l'informatique, il décrit une méthode \Gls{g-agile} qui a la particularité d'organiser le travail sous forme d'échéances à court terme que l'on appelle les sprints\cite{noauthor_guide_2013}}
}
\newglossaryentry{g-api}{
    name={application programming interface},
    description={que l'on traduit interface de programmation applicative, est un ensemble normalisé de classes, de méthodes ou de fonctions qui sert de façade par laquelle un logiciel offre des services à d'autres logiciels\cite{wikipedia_interface_2019}}
}
\newglossaryentry{g-log}{
    name={log},
    description={est dans le contexte de l'informatique le mot anglais pour le journal d'application. Ce document est rempli par le programme auquel il est lié et permet aux développeurs de comprendre les événements s'étant déroulés au sein de sa vie, de la même manière que le journal de bord d'un capitaine de navire}
}
\newglossaryentry{g-pma}{
    name={PhpMyAdmin},
    description={est une application web très populaire pour administrer une base de données}
}
\newglossaryentry{g-crud}{
    name={Create Retrieve Update Delete},
    description={sont les quatre opérations fondamentale de n'importe quelle base de données. Elles permettent de créer de nouveaux enregistrements (\textit{Create}), les récupérer (\textit{Retrieve}), les modifier (\textit{Update}) et les supprimer (\textit{Delete})}
}
\newglossaryentry{g-jhipster}{
    name={Jhipster},
    description={est un générateur de code source. Il produit du code à la place du développeur pour les besoins classiques de celui-ci en matière de site web \cite{noauthor_jhipster_nodate}}
}
\newglossaryentry{g-orm}{
    name={object relational mapping},
    description={est un composant logiciel qui interface une base de données relationnelle et un programme qui utilise cette dernière}
}
\newglossaryentry{g-hibernate}{
    name={Hibernate},
    description={est un \gls{a-orm} très populaire dans le monde \gls{g-java}\cite{noauthor_hibernate._nodate}}
}
\newglossaryentry{g-nas}{
    name={network attached storage},
    description={que l'on traduit \gls{g-server} de stockage en réseau, permet le partage de fichiers à tous les \glspl{g-client} du réseau en un volume centralisé\cite{wikipedia_serveur_2018}}
}
\newglossaryentry{g-mysql}{
    name={MySQL},
    description={est un système de gestion de bases de données relationnelles\cite{wikipedia_mysql_2018} parmi les plus populaires au monde\cite{solid_db-engines_2019}}
}
\newglossaryentry{g-java}{
    name={Java},
    description={est un langage de programmation très populaire dans les entreprises permettant de créer des programmes compatible avec de nombreux appareils électroniques}
}
\newglossaryentry{g-spring}{
    name={Spring},
    description={est un socle d'application \gls{g-java} pour créer toutes sortes d'applications}
}
\newglossaryentry{g-angular}{
    name={Angular},
    description={est un socle d'application \gls{g-javascript} pour créer des applications web qui s'exécutent sur le terminal de l'utilisateur}
}
\newglossaryentry{g-javascript}{
    name={JavaScript},
    description={est un langage de programmation initialement conçu pour s'exécuter sur les navigateurs internet et dont l'usage s'est répandu à d'autres domaines. C'est le langage de programmation le plus populaire au monde\cite{noauthor_brief_nodate}}
}
\newglossaryentry{g-mapping}{
    name={mapping},
    description={est l'action de transférer les données d'une structure à une autre}
}
\newglossaryentry{g-sql}{
    name={structured query language},
    description={est un language de programmation permettant de stocker, manipuler et récupérer des données dans une base de données\cite{noauthor_sql_nodate}}
}
\newglossaryentry{g-http}{
    name={Hypertext Transfer Protocol},
    description={est un standard du web sur la manière de communiquer à travers un réseau, peu importe son support physique}
}
\newglossaryentry{g-url}{
    name={Uniform Resource Locator},
    description={est un standard du web pour localiser une ressource}
}
