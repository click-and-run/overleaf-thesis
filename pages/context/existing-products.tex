\section{Les produits existants}
\label{sec:existing-products}

\paragraph{}
Le coeur du projet consiste à implémenter les règles logiques qui régissent le contenu de cours et de tests d'Altissia.
Il n'existe donc aucun produit existant qui fasse cela.

\paragraph{}
Toutefois, il est possible de choisir de s'appuyer sur certains produits existants pour une partie des fonctionnalités.
Il y a notamment une partie de l'application qui doit gérer l'exécution des différentes tâches, reporter leur statut, tenir un historique, etc.
Ces responsabilités tombent sous le coup d'un planificateur de tâches.
% Est-ce que je peux uniquement utiliser le front de Rundeck ou dois-je créer un front intermédiaire qui appelle l'API de rundeck ?

\paragraph{}
Des exemples connus sont: Ansible, Rundeck, Salt, Chef et Fabric.
Il se trouve que Altissia utilise justement Rundeck pour automatiser ses déploiements\fnmark. Il n'y a donc pas lieu d'acheter des licences pour une alternative ou de s'y former.

\fntext{Dans le contexte du développement informatique, le déploiement est l'acte de rendre une application disponible aux utilisateurs.}

\paragraph{}
Rundeck se présente sur sa page d'accueil\cite{rundeck_rundeck_nodate}, traduit de l'anglais:
\begin{displayquote}
Transformez vos procédure opérationnelles en tâches disponibles en libre service. Donnez le contrôle et la transparence aux autres en toute sécurité.
\end{displayquote}
Ce qui est bien le but recherché.
Rendre disponible des processus existants et donner le contrôle aux acteurs métiers. 
