\chapter{Introduction}
\label{ch:introduction}

\paragraph{}
Ces quatre dernières années, j'ai suivi les études pour acquérir le bachelier en informatique de gestion.
Le développement informatique me donne l'impression que tout peut être automatisé.
Ce présent travail est un argument en faveur de cette assertion et un formidable défi aux compétences et connaissances acquises lors de ma formation.

\paragraph{}
Je travaille à Altissia depuis deux ans.
Je développe de nouvelles fonctionnalités, corrige des bogues et réalise différentes tâches opérationnelles.

\paragraph{}
A Altissia, ainsi que dans de nombreuses entreprises, un grand nombre de données sont manipulées.
Ces données peuvent être celles des clients, des utilisateurs, des métiers, etc.
Et pour de nombreux employés, la méthode la plus simple pour stocker ces données est le tableur et bien souvent, le tableur Excel.

Malheureusement, il arrive fréquemment que ces données soient mal formatées ou invalides et ne permettent pas leur exploitation par les logiciels cibles.
Il incombe alors au service informatique de régler ces soucis, car ils ont de meilleures connaissances du format requis et disposent des meilleurs outils pour les corriger.

\paragraph{}
À l'heure actuelle, nous utilisons un programme appelé \textit{Altissia-launcher}.
Ce programme valide les données contenues dans des classeurs Excel, il les manipule et les transforme afin qu'elles puissent être exploitées par d'autres logiciels de l'entreprise.

Malheureusement, cet outil est très loin d'être parfait.
Il n'est utilisable que par un développeur, car il ne dispose pas d'interface graphique; on l'exécute directement depuis l'\gls{a-edi}.
De plus, même pour le développeur, il est difficile d'utilisation, car pour mener à bien un processus complet, il reste nécessaire d'appliquer manuellement certaines tâches.
Il faut par exemple, déplacer ou copier des fichiers d'un endroit à l'autre, lancer plusieurs scripts dans un certain ordre, etc.
En fait, il est impératif de suivre une documentation\fnmark pas à pas sous peine de se tromper et de devoir recommencer.
Enfin, il utilise des technologies désuètes qui le rendent très difficile à maintenir ou à faire évoluer.

\fntext{Vous pouvez trouver la documentation d'un de ces processus en annexe \ref{ch:altissia-launcher-doc}.}

\paragraph{}
Grégory, le directeur technique d'Altissia, m'a proposé de développer un nouvel outil qui aura pour but de supplanter \textit{Altissia-launcher}.

\paragraph{}
L'objectif de ce nouveau logiciel en plus de reprendre les fonctionnalités de son prédécesseur sera de proposer une interface graphique ergonomique afin qu'il puisse être utilisé par un acteur métier plutôt que par un développeur.
\textbf{En fait, plutôt que d'être une solution concrète, cela sera un cadre de travail proposant aux développeurs les outils pour pouvoir implémenter tous les cas possibles et imaginables.}
