\section{Le socle d'application Click-and-Run}
\label{sec:spreadsheet-framework}

\paragraph{}
Les outils mis en place doivent au minimum permettre de résoudre les cas d'utilisation proposés.
Mais ils doivent aussi être le plus générique possible afin de pouvoir être appliqués aux plus de cas possibles.
Le socle d'application doit être conçu de manière à ce qu'il soit le plus facile possible d'intégrer de nouveaux outils.

\paragraph{}
Le but de créer un socle d'application plutôt qu'une application concrète est d'être capable de gérer de nouveaux cas avec un temps de développement minimal.
Il est créé libre de droits afin que n'importe qui puisse s'en servir et y contribuer.
Il est donc logique qu'il contienne des outils prêts à l'utilisation, qu'il soit prêt à accueillir des outils inconnus et qu'il soit compatible avec des standards informatiques répandus.

\paragraph{}
Je dois créditer le projet Poiji pour m'avoir fait gagner beaucoup de temps.
C'est une petite librairie \Gls{g-java} qui permet de faire le \gls{g-mapping} entre les données contenues dans un fichier Excel et un objet \Gls{g-java}\cite{ozler_:candy:_2019}.

\subsection{Le fonctionnement de Click-and-Run}
\label{subsec:operation}

\paragraph{}
Le but de la validation est d'apporter le plus d'informations possibles à l'\textbf{utilisateur} afin qu'il puisse corriger lui-même classeur qu'il a soumis.
De manière générale, le vérificateur va donc essayer d'en trouver le plus possible et ne s'arrêtera pas au premier cas rencontré.
Il y a néanmoins certains cas d'erreurs qu'il faut envisager différemment.

\subsubsection{La vérification des entêtes ou du contenu}
\label{subsubsec:headers-validation}

\paragraph{}
Lorsqu'une feuille est manquante ou que les entêtes ne sont pas valides, son contenu n'est pas validé.
Cela relèverait trop d'erreurs qui ne feront pas forcément sens.
Par exemple, si l'entête d'un champs obligatoire est manquante, chaque ligne va déclaré le champs comme manquant.

De plus, les validations propres aux feuilles plutôt qu'au contenu ne seront pas exécutés non plus.
Cela concerne par exemple la vérification des contraintes de référence.
Dans notre cas d'utilisation, si la feuille des apprenants est vide, il ne sera pas possible de vérifier qu'un service est lié à un élève défini dans cette feuille.
Mais la feuille des services peut par contre valider ses règles propres.

\paragraph{}
Il n'est donc pas possible d'avoir à la fois des erreurs dans les entêtes et dans le contenu car le premier empêche le deuxième.

\subsubsection{La vérification par lot}
\label{subsubsec:batch-validation}

\paragraph{}
Dans les cas où la vérification d'une règle nécessite d'accéder à une ressource externe, comme un \gls{a-api} ou une base de données,
elle se fera à la fin de la vérification de la feuille et une seule fois plutôt qu'une fois par ligne afin de réduire la lenteur qu'engendrerait ces multiples appels.

Lorsque l'on vérifie qu'un email n'est pas déjà utilisé dans la base de données, on pourrait vérifier ligne par ligne.
Ce serait très inefficace car un appel à une ressource externe est très lent, comparé à l'exécution du code qui ne nécessite par d'appels externes.
Il est bien plus optimal, à la fois pour la vérification et pour le service externe concernée, de faire un seul appel qui livre toutes informations requises en une seule fois.

\paragraph{}
Pour la vérification de l'existence des emails, le programme devra donc constituer la liste complète, vérifier leur présence en base de données en une seule requête et mettre en erreur toutes les lignes liées aux emails qui existaient.

\paragraph{}
Un autre cas concerné problématique lors d'une approche ligne par ligne est la contrainte d'unicité.
En effet, pour vérifier qu'une valeur est unique, il est préférable de le faire avec la liste complète plutôt que de le faire au fur et à mesure.

C'est un problème de complexité temporelle.
La complexité temporelle exprime le rapport entre le temps d'exécution d'un programme et la taille des données qu'il traite.
Ici, la taille du problème est le nombre de ligne, que l'on va appeler $n$.
Dans le cas où l'on vérifie la contrainte à chaque ligne, on doit considérer toutes les lignes précédentes pour chaque ligne.
Le calcul du temps d'exécution est donc le suivant:
\begin{align}
line_1&= 1 \\
line_2&= 2 \\
line_3&= 3 \\
line_n&= n \\
\sum_1^n &= 1+2+3+...+n \\
\sum_1^n &= \frac{n^2 + n}{2}
\end{align}
Il n'a pas d'unité à ce calcul, la seule notion importante est la fonction appliquée à $n$.
Et on ne retient que la fonction qui évolue le plus rapidement.
C'est donc le carré de $n$ qui est retenu.
On note cette complexité temporelle $O(n^2)$, qui veut dire quand dans le pire des cas, le temps d'exécution du programme est proportionnel au carré de $n$.

Dans le cas où l'on traite toutes les lignes en une seule fois, on doit constituer une liste triée, parcourir cette liste pour calculer le nombre d'apparition de cette valeur dans la liste.
Trier la liste permet de retrouver très rapidement n'importe quelle valeur dans cette liste.
Trier la liste a une complexité $O(log(n))$ et lire un élément aléatoire de cette liste a une complexité $O(1)$ qui exprime un temps constant, non dépendant du nombre d'éléments.
Il faut lire $n$ éléments une fois ce qui donne une complexité $O(n)$.

L'opération du tri et du parcours de la liste sont faites séquentiellement et leur complexité doivent donc s'additionner.
Dans le cas d'une addition, seule la fonction la plus forte est gardée, c'est à dire $O(n) + O(log(n)) = O(n)$.

Vérifier l'unicité par lot plutôt que par ligne permet donc de réduire la complexité temporelle de $O(n^2)$ à $O(n)$.

\subsubsection{La vérification entre feuilles}
Vérifier qu'un email défini dans la feuille des apprenants est utilisé au moins une fois dans la liste des services et qu'un email utilisé dans la feuille des services est défini dans la feuille des apprenants nécessite une autre sorte de vérification.

Il faut mettre en place une vérification qui dispose de l'entièreté du classeur.

\subsection{Les outils prêts à l'emploi}
\label{subsec:ready-tools}

\subsection{Les outils à intégrer}
\label{subsec:host-tools}

\paragraph{}
Certaines règles sont trop spécifiques et il n'est pas possible de les généraliser.
Il est par contre possible de simplifier la vie du développeur qui va devoir les ajouter en prévoyant un système qui va détecter la règle et l'appliquer au bon moment et dans les bons cas.

\paragraph{}
La personne qui veut rajouter une règle dispose de trois points d'entrée:
\begin{itemize}
    \item Le modèle d'une feuille: elle peut associer sa règle au modèle qui représente une ligne
    \item La validation de feuille: elle peut ajouter un service qui sera appelé lors de la validation de la feuille
    \item la validation de classeur: elle peut ajouter un service qui sera appelé lors de la validation du classeur
\end{itemize}
Afin que cela fonctionne, elle doit respecter les formats prévus par mon application.

\paragraph{}
La règle d'un modèle est appliquée sous la forme d'une annotation \textit{javax.validation.constraints}.
C'est un format standard pour la validation que je discute dans la section suivante.

\paragraph{}
Pour la validation d'une feuille ou d'un classeur, le développeur doit créer un service qui doit:
\begin{enumerate}
    \item Indiquer qu'il est capable de traiter le type de données que la validation est en train de vérifier
    \item Implémenter une méthode avec une signature précise qui effectue la vérification
\end{enumerate}
Ce mécanisme correspond au patron de conception de la stratégie.
Il est discuté dans la section \ref{subsec:strategy-pattern}.

\subsection{Les outils compatibles}
\label{subsec:compatible-tools}

\paragraph{}
Pour valider une ligne d'un fichier Excel, j'ai choisi de procéder en deux étapes:
\begin{itemize}
    \item Transférer les données de la ligne sur un modèle d'objet
    \item Valider l'objet
\end{itemize}
C'est une stratégie qui très usitée pour la validation de données avant de les stocker dans une base de données ou encore lors de réception de données d'un utilisateur.
L'objet est une abstraction qui permet de représenter de manière intuitive les données sans tenir compte de l'implémentation sous-jacente du stockage utilisé.

\paragraph{}
J'ai choisi d'utiliser un format établi qui est utilisé par de nombreux outils existants sur le marché: \textit{javax.validation.constraints}.
Ce choix permettra instantanément de disposer de centaines de règles déjà écrites et utilisées dans l'industrie.

Dans l'extrait de code suivant, on peut observer à quoi ressemble un modèle et observer quelques annotations.
\lstinputlisting[language=java, firstnumber=14, firstline=14, lastline=37]{pages/appendix/code/RegistrantRow.java}
Les annotations \lstinline{@NotNull} et \lstinline{@Pattern} sont directement définies par la librairie \textit{javax.validation.constraints}.
Les annotations \lstinline{@Email} et \lstinline{@Length} viennent de l'\gls{a-orm} \Gls{g-hibernate}.
L'annotation \lstinline{@InterfaceLanguage} a été créée par mes soins.
Les autres annotations ne servent pas à valider les données mais à faire le lien entre la ligne dans le fichier Excel et le modèle d'objet.

\subsection{Le diagramme de classe}
\label{subsec:class-diagram}

\begin{figure}[ht]
    \centering
    \includegraphics[width=1\textwidth]{images/diagrams/clickandrun-class-diagram.png}
    \caption{Le diagramme complet contenant les classes nécessaires à la validation et au traitement des données}
    \label{fig:class-diagram-full}
\end{figure}

\paragraph{}
Dans le diagramme \ref{fig:class-diagram-full}, on peut observer différents blocs.
Avant de passer en revue ces différents blocs, j'aimerais attirer l'attention sur la classe \lstinline{WorkbookExtendedService}.
Cette classe est de très loin la plus importante et son code est disponible en annexe \ref{ch:workbook-service-code}.

\begin{figure}[ht]
    \centering
    \includegraphics[width=1\textwidth]{images/diagrams/workbook-service-class.png}
    \caption{La classe qui connecte tous les bouts}
    \label{fig:class-service}
\end{figure}

\paragraph{}
Bien que cette classe soit le coeur de l'application, elle n'étend aucune classe, n'implémente aucune interface et n'est étendue par aucune classe.
Cela est possible grâce à quatre choses:
\begin{itemize}
    \item Les variables \lstinline{sheetValidators}, \lstinline{workbookValidators} et \lstinline{processors} sont toutes trois injectées par \Gls{g-spring} (voir sous-section \ref{subsubsec:dependency-injection} pour plus de détails sur l'injection de dépendances)
    \item La classe définit un paramètre générique \lstinline{T} qui peut être n'importe quelle classe qui étend la classe \lstinline{Row}
    \item Les instances de la classe \lstinline{T} sont peuplées en utilisant le mécanisme de réflexion\fnmark{}
    \fntext{Le mécanisme de réflexion est la capacité qu'un programme a à s'examiner lui-même.}
    \item Le patron de conception de la stratégie permet d'obtenir la stratégie adaptée pour traiter un objet sans connaitre cet objet.
\end{itemize}

Un paramètre générique permet d'utiliser une classe non spécifiée tout en étant capable d'utiliser ses capacités spécifiques.
Lors de l'exécution du code, \lstinline{T} pourrait en réalité être \lstinline{RegistrationWorkbook}.
La généricité n'est pas à confondre avec le polymorphisme\fnmark{}.
\fntext{Le polymorphisme est la capacité d'une fonction informatique à accepter tout paramètre qui est descendant du type attendu. Ainsi, si vous avez besoin d'un fruit, vous serez satisfait d'une banane mais vous ne pourrez pas vous servir des caractéristiques propres à la banane car vous ne savez pas que c'est une banane. Vous savez juste que c'est un fruit.}
Ce dernier ne permet pas d'accéder au comportement spécifique de l'instance manipulée.

Le mécanisme de réflexion permet d'examiner la classe non spécifiée pour comprendre son comportement.
En examinant la classe \lstinline{RegistrationWorkbook}, le programme détermine quels champs correspondent à quelles données du fichier Excel.

La patron de conception de la stratégie permet de choisir les stratégies adaptés à l'instance traitée.
Le service a reçu toutes les stratégies possibles lors de sa création.
Ces stratégies ne sont autres que les membres des variables \lstinline{sheetValidators}, \lstinline{workbookValidators} et \lstinline{processors}.
Il passe en revue toutes les stratégies une à une et elle détermine elle-même si elles sont adaptées ou non.
C'est le concept caché dernière ce patron de conception; c'est la stratégie qui définit si elle est valide ou non, et pas le service qui emploie la stratégie.

Pour être une stratégie, une classe doit implémentée une interface qui fournit deux méthodes: une qui valide la compatibilité de la stratégie et une qui applique la stratégie.
Voici l'interface d'un des trois types de stratégie:
\begin{lstlisting}[language=Java]
/**
 * Interface to define a post-validator for business / functional validation in the whole workbook
 */
public abstract class WorkbookValidator {

 protected static final Logger log = LoggerFactory.getLogger(WorkbookValidator.class);
 
 /**
  * Assert if this validator is suitable for a particular workbook
  * @param definition the definition to check
  * @return true if this validator is suitable for the provided workbook
  */
 public abstract boolean isApplicableTo(Workbook definition);
 
 /**
  * Actual validation process
  * @param workbook the workbook to add error / warning to
  * @return the number of error/warning issued by this validator
  */
 public abstract long validate(Workbook workbook);

}
\end{lstlisting}

\begin{figure}[ht]
    \centering
    \includegraphics[width=1\textwidth]{images/diagrams/row-dependencies.png}
    \caption{Les classes liées au modèle}
    \label{fig:class-model}
\end{figure}
\paragraph{}
Le bloc repris dans la figure \ref{fig:class-model} est nettement orienté.
Tous les flèches sont dirigées vers les classes modèles du \gls{g-framework}.

Les cinq classes \lstinline{RegistrationWorkbook}, \lstinline{ServicesSheet}, \lstinline{RegistrantSheet}, \lstinline{ServiceRow} et \lstinline{RegistrantRow} constituent un module.
L'ajout d'un module nécessite l'écriture ces classes-là, plus éventuellement les stratégies non prévues par le \gls{g-framework}.

\paragraph{}
Les classes restantes n'apportent rien de fondamentalement différent de ce qui a été abordé ici et je ne les passerai donc pas en revue.


