\subsection{La page de validation et traitement des classeurs}
\label{subsec:spreadsheet-page}

\paragraph{}
La page de validation et traitement des données est la page centrale de l'application et c'est elle qui va justifier le plus de travail.
Son fonctionnement doit être le plus simple possible.
Un \textbf{utilisateur} doit pouvoir s'en servir sans formation et ses actions ne peuvent pas tromper ses intentions.

\paragraph{}
La page est divisée en trois éléments:
\begin{itemize}
    \item Le récepteur de fichier: il permet de choisir le fichier que l'on valide
    \item Le tableau des résultats: il affiche les résultats de la validation
    \item Le bouton de traitement: il lance le traitement du fichier validé
\end{itemize}

\paragraph{}
Le récepteur de fichier permet de choisir un fichier de deux manières, soit en cliquant dessus, soit en glissant et en déposant un fichier sur lui.
Dans le premier cas, il ouvre le menu système\fnmark{} de l'utilisateur. Dans le second cas, l'utilisateur a cliqué sur le fichier et a maintenu le clic, a glissé le fichier au-dessus du composant et a relâché le clic.
\fntext{Le menu système est le menu prévu par le système d'exploitation installé sur la machine de l'utilisateur}

Dans le cas où le fichier n'est pas compatible, car il n'est pas un fichier Excel ou est corrompu, un message d'erreur renseigne l'utilisateur.
Dans le cas où le fichier est compatible, la validation de son contenu est immédiatement lancée sans autre action de l'utilisateur.

Le nom du fichier sélectionné avec succès est affiché sur le récepteur.

Un double-clic sur le récepteur vide le fichier.

Des instructions indiquent les actions à prendre en fonction de l'état du fichier: présent ou non présent. % TODO gérer valide et non valide ?

\paragraph{}
Le tableau des résultats affiche les erreurs de validation relevées par le \gls{g-server}.
Le résultat de chaque feuille est affiché dans son propre onglet.
Les erreurs sont divisées en trois tableaux distincts: les problèmes de titre, les erreurs de contenu et les avertissements sur le contenu.
Pour les problèmes de titres, le numéro de colonne, le titre de la colonne, la valeur dans le fichier pour cette colonne et le message d'erreur sont affichés.
Le message d'erreur est exprimé dans un langage intelligible par un \textbf{utilisateur}.
Les mêmes champs sont utilisés par les erreurs et avertissements si ce n'est que la colonne devient une ligne.
Les erreurs sont affichées dans l'ordre des lignes.

\paragraph{}
Le bouton de traitement n'est utilisable que si le fichier a passé la validation.
Son activation envoie le fichier déjà sélectionné et déjà validé pour son traitement.
En cas de succès, ce qui est sensé toujours être le cas, un message de succès apparait et le fichier est vidé du récepteur de fichier.
