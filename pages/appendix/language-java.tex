\chapter{La classe \textit{Language.java}}
\label{ch:language-java}

\paragraph{}
Cette classe est une reproduction simplifiée d'une classe utilisée par les services d'Altissia pour manipuler la notion de langue.
L'intérêt est de constater que ce n'est pas une simple liste de langues mais qu'elle définit une nomenclature stricte et classe ses sujets selon les critères d'utilisation.
Toutes langues listée ici est une langue d'interface et certaines sont des langues d'étude.

\paragraph{}
Cela se comprends car il est beaucoup plus difficile de créer le contenu d'apprentissage que traduire les textes des interfaces.
De plus, afficher du contenu de cours dans une langue nécessite d'avoir fait le travail requis pour afficher cette langue.
Il ne reste donc en comparaison aucun effort à faire pour transformer une langue d'étude en une langue d'interface.
Et cet effort est systématiquement fait.

\lstinputlisting[language=java]{pages/appendix/Language.java}