\subsection{Les demandes clients}
\label{subsec:customer-requests}

Voici les différentes fonctionnalités établies au cours de l'élaboration du cahier des charges:
\begin{enumerate}
    \item Valider de fichiers Excel:
    \begin{enumerate}
        \item Présence des feuilles
        \item Présence des entêtes
        \item Ignorer les entêtes superflus
        \item Le respect d'un modèle de chaine de caractères (une \gls{g-regex}).
        \item La position d'une valeur numérique par rapport à une borne inférieure
        \item La position d'une valeur numérique par rapport à une borne supérieure
        \item L'inclusion d'une valeur numérique entre deux bornes
        \item Le résultat de l'évaluation d'une \gls{g-bool-func} arbitraire
        \item La taille d'une chaine de caractères
        \item L'appartenance d'une valeur à un ensemble fini
        \item La contrainte de référence\fnmark{} entre deux feuilles du même classeur
        \fntext{Une contrainte de référence signifie qu'une valeur doit exister dans le référentiel. Par exemple, mon nom de famille doit être porté par l'un de mes parents pour que je puisse le porter.}
        \item La contrainte de référence dans une base de données
        \item La contrainte de référence dans un API externe
        \item L'unicité d'une valeur au sein de sa colonne
        \item L'unicité d'une combinaison de valeurs au sein de sa feuille
        \item Validation d'une cellule en fonction de la valeur d'une autre cellule
        \item Le \gls{g-null}
        \item Le respect du format des \textit{mails}
    \end{enumerate}
    \item Contrôler l'application depuis une interface web:
    \begin{enumerate}
        \item Soumettre un fichier pour validation
        \item Visualiser la validation
        \item Soumettre un fichier pour exécution
        \item Visualiser le résultat de l'exécution
        \item Visualiser l'historique des exécutions
        \item Être notifié de la complétion d'une validation ou exécution
    \end{enumerate}
\end{enumerate}

Ce sont des fonctionnalités qui répondent à un cas concret.
Mon but est de créer un outil générique capable de répondre à ces besoins.
