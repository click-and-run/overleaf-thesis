\chapter{Extrait du code d'Altissia launcher}
\label{ch:altissia-launcher-code}

\paragraph{}
Lorsque l'on recycle du code, il ne s'agit pas tellement de copier coller le code existant, mais plutôt d'en extraire la logique.

\paragraph{}
Ce premier extrait de code vient de la définition du \textit{build}\fnmark{} Apache Ant.
\fntext{En informatique, le \textit{build} est la phase qui consiste à transformer le code source en logiciel fonctionnel.}
Il permet de définir le fonctionnement d'un script à un haut niveau d'abstraction.

\lstinputlisting[language=XML, firstnumber=1521]{pages/appendix/altissia-launcher-build.xml}

\paragraph{}
Un deuxième extrait de code qui contient le plus gros de la logique de validation d'une question de test de niveau.
Un développeur expérimenté devrait être capable de trouver quelques points à améliorer dans ce code.
J'ai aussi enlevé quelque 200 lignes de code par souci de confidentialité.

\lstinputlisting[language=java]{pages/appendix/altissia-launcher-validate-excel-question.java}
