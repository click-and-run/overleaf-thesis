\section{Le socle d'application de validation et de traitement des classeurs}
\label{sec:spreadsheet-framework}

\paragraph{}
Les outils mis en place doivent au minimum permettre de résoudre les cas d'utilisation proposés.
Mais ils doivent aussi être le plus générique possible afin de pouvoir être appliqués à le plus de cas possibles.
Le socle d'application doit être conçu de manière à ce qu'il soit le plus facile possible d'intégrer de nouveaux outils.

\paragraph{}
Le but de créer un socle d'application plutôt qu'une application concrète est d'être capable de gérer des nouveaux cas avec un temps de développement minimale.
Il est créé libre de droits afin que n'importe qui puisse s'en servir et y contribuer.
Il est donc logique qu'il contienne des outils prêts à l'utilisation, qu'il soit prêt à accueillir des outils inconnus et qu'il soit compatible avec des standards informatiques répandus.

\subsection{Le fonctionnement de Click-and-Run}
\label{subsec:operation}


\subsection{Les outils prêts à l'emploi}
\label{subsec:ready-tools}

\paragraph{}
La plupart des règles de validation ne sont pas propres à un seul classeur, il faut donc les concevoir de façon à ce qu'elles soient utilisables dans d'autres classeurs.

\paragraph{}
Je ne vais donc pas implémenter une règle pour vérifier un format de chaine de caractère précis, mais une règle pour vérifier les formats de manière général.

Je n'ai pas créé une règle qui vérifie qu'un nombre est compris entre 0 et 365, mais deux règles: une pour vérifier qu'une valeur est plus grande qu'une borne et une qui vérifie que la même valeur est inférieure à une autre borne.

Vérifier qu'une valeur fait parti de la liste des services revient à vérifier qu'une valeur est un élément d'un ensemble.

Vérifier que le nom est long de deux à cinquante caractères résulte en une règle qui vérifie la taille d'une chaine de caractères.

Vérifier que l'email est bien défini revient à vérifier qu'il n'est pas \gls{g-null}.


\subsection{Des outils à intégrer}
\label{subsec:host-tools}

\paragraph{}
Certaines règles sont trop spécifiques et il n'est pas possible de les généraliser.
Il est par contre possible de simplifier la vie du développeur qui va devoir les ajouter en prévoyant un système qui va détecter la règle et l'appliquer au bon moment et dans les bons cas.

\paragraph{}
La personne qui veut rajouter une règle dispose de trois points d'entrée:
\begin{itemize}
    \item Le modèle d'une feuille: elle peut associer sa règle au modèle qui représente une ligne
    \item La validation de feuille: elle peut ajouter un service qui sera appelé lors de la validation de la feuille
    \item la validation de classeur: elle peut ajouter un service qui sera appelé lors de la validation du classeur
\end{itemize}
Afin que cela fonctionne, elle doit respecter les formats prévus par mon application.

\paragraph{}
La règle d'un modèle est appliquée sous la forme d'une annotation \textit{javax.validation.constraints}.
C'est un format standard pour la validation que je discute dans la section suivante.

\paragraph{}
Pour la validation d'une feuille ou d'un classeur, le développeur doit créer un service qui doit:
\begin{enumerate}
    \item Indiquer qu'il est capable de traiter le type de données que la validation est en train de vérifier
    \item Implémenter une méthode avec une signature précise qui effectue la vérification
\end{enumerate}
Ce mécanisme correspond au patron de conception de la stratégie.
Il est discuté dans la section \ref{subsec:class-diagram}.


\subsection{Les outils compatibles}
\label{subsec:compatible-tools}

