\subsection{Les pages de l'administrateur}
\label{subsec:admin-pages}

\paragraph{}
L'\textbf{administrateur} a accès à toute une panoplie d'outils.
Ces pages sont accessibles par un menu qui n'est visible que pour l'\textbf{administrateur}.

Les éléments suivants doivent être présents:
\begin{itemize}
    \item Gateway: Liste les API disponibles
    \item Gestion des utilisateurs: \acrshort{a-crud} sur les listes des \textbf{utilisateurs}
    \item Métriques: Mesures de l'utilisation des ressources de la machine physique sur laquelle est installée l'application
    \item Diagnostics: Liste des services composants l'application et leur état de fonctionnement
    \item Configuration: Liste des propriétés configurables du serveur et leur valeur
    \item Audits: Liste des appels sur les ressources surveillées (ex.: tentative de connexion)
    \item \Glspl{g-log}: Contrôle des niveaux de verbosité des agents de journaux du \gls{g-server}
    \item \gls{a-api}: Liste des ressources exposées par le \gls{g-server} avec exemples d'utilisation
    \item Base de données: Lien vers l'application d'accès à la base de données (ex.: \gls{g-pma})
\end{itemize}

Ces pages doivent correspondre exactement à celles fournies par le générateur de code \Gls{g-jhipster}.
Cet outil est utilisé par Altissia et permet de gagner plusieurs de développements lors de la création de nouveaux projets.
Il génère le code qui constitue le squelette d'un site web et le développeur se l'approprie pour ajouter de nouvelles fonctionnalités par dessus.
