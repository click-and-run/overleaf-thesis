\chapter{Extrait du code d'Altissia launcher}
\label{ch:altissia-launcher-code}

\paragraph{}
Lorsque l'on recycle du code, il ne s'agit pas tellement de copier coller le code existant mais plutôt d'en extraire la logique.

\paragraph{}
Ce premier extrait de code vient de la définition du \textit{build}\fnmark Apache Ant.
Il permet de définir le fonctionnement d'un script à un haut niveau d'abstraction.
C'est typiquement ce que je remplace avec Rundeck.

\fntext{En informatique, le \textit{build} est la phase qui consiste à transformer le code source en logiciel fonctionnel.}

\lstinputlisting[language=XML, firstnumber=1521]{pages/appendix/altissia-launcher-build.xml}

\paragraph{}
Un deuxième extrait de code qui contient le plus gros de la logique de validation d'une question de test de niveau.
Si vous avez un peu d'expérience en programmation, vous devriez pouvoir trouver quelques erreurs...
J'ai aussi enlevé quelques 200 lignes de code par soucis de confidentialité.

\lstinputlisting[language=java]{pages/appendix/altissia-launcher-validate-excel-question.java}
