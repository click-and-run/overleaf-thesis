\subsection{Le serveur avec Spring}
\label{subsec:server-spring}

\paragraph{}
Le \gls{g-server} est implémenté en utilisant le socle d'application \Gls{g-spring} Boot.
C'est une version de \gls{g-spring} où des choix dogmatiques ont été faits sur la manière dont \gls{g-spring} doit être configuré afin de réduire le travail nécessaire à avoir un \gls{g-server} prêt l'emploi.
\Gls{g-spring} est un projet absolument immense et simplifie la vie du développeur dans nombreux contextes.
Je vais me concentrer ici sur quatre aspects:
\begin{itemize}
    \item La base de données
    \item Le réseau
    \item La sécurité
    \item L'injection de dépendance
\end{itemize}

\subsubsection{La base de données}
\label{subsubsec:spring-data-jpa}

\paragraph{}
Son composant \gls{g-spring} Data \acrshort{a-jpa}, qui est bati par-dessus le projet \gls{g-hibernate}, gère la connexion a la base de données, la représentation des données sous forme d'objet ainsi que la persistance des données.

Ainsi un développeur n'a plus besoin d'écrire la moindre requête \gls{a-sql}.
Il utilise les fonctions fournies par \gls{g-spring} Data \acrshort{a-jpa}.

On peut par exemple voir ci-dessous la classe qui représente le répertoire dans lequel sont stockée les entitées des apprenants:
\begin{lstlisting}[language=Java]
@Repository
public interface LearnerRepository extends JpaRepository<Learner,Long> {
    List<Learner> findAllByLoginIn(Collection<String> loginList);
}
\end{lstlisting}
La méthode \lstinline{findAllByLoginIn} n'est pas prévue de base mais est construire en utilisant les outils fournis par le \gls{g-framework}.

L'appel \lstinline{learnerRepository.save(learners);} utilise une fonction prévue pour sauvegarder plusieurs apprenants en base de données.
A partir de là, \gls{g-spring} Data \acrshort{a-jpa} s'occupe de créer la requête \gls{a-sql} correspondant.

\subsubsection{Le réseau}
\label{subsubsec:spring-web}

\paragraph{}
Le composant \Gls{g-spring} Web est sûrement le plus utilisé de \gls{g-spring}.
Il permet de transformer les données reçus par le réseau en objet défini par le code java.
En cas de tout écart au scénario nominal, il s'occupe de répondre avec le bon code d'erreur à l'appelant.
Il va par exemple répondre "404 \textit{Not Found}" lorsque la ressource recherché n'existe pas ou encore "500 \textit{Internal Server Error}" lorsque l'application plante et n'est plus capable de formuler une réponse correcte.

\paragraph{}
Voici par exemple une ressource définie par l'\gls{a-api} de Click-and-Run\fnmark{}:
\fntext{J'ai caché certains détails d'implémentation pour simplifier.}
\begin{lstlisting}[language=Java]
@Timed
@PostMapping(value = "/api/registration/validate", produces = MediaType.APPLICATION_JSON_UTF8_VALUE)
public ResponseEntity<Workbook> validateFile(@RequestParam(name = "file") MultipartFile file) {
    log.debug("REST request to /registration/validate with {}", file.getOriginalFilename());

    Workbook workbook = workbookExtendedService.validateWorkbook(file, new RegistrationWorkbook());

    return ResponseEntity.ok(workbook);
}
\end{lstlisting}
Ce code ne contient que très peu de logique et fait surtout appel aux fonctionnalités de \Gls{g-spring} Web. Il indique à quelle URL la ressource doit répondre, avec quel verbe \gls{a-http}, avec quel paramètre et avec quelle réponse.

\subsubsection{La sécurité}
\label{spring-boot-starter-security}

\paragraph{}
Le composant \gls{g-spring} Security s'occupe de fournir des mécanismes de sécurité pour protéger les différentes ressources du serveur avec un degré de granularité personnalisable.
Il vérifie que chaque demande est légitime.

Pour personnaliser ma configuration du \gls{g-server}, j'ai écris la fonction suivante dans la classe \lstinline{WebSecurityConfigurerAdapter} qui est utilisée par le \gls{g-framework} pour se configurer.
\begin{lstlisting}[language=Java]
@Override
protected void configure(HttpSecurity http) throws Exception {
    http
        .csrf()
        .disable()
        .headers()
        .frameOptions()
        .disable()
    .and()
        .sessionManagement()
        .sessionCreationPolicy(SessionCreationPolicy.STATELESS)
    .and()
        .authorizeRequests()
        .antMatchers("/api/**").authenticated()
        .antMatchers("/management/health").permitAll()
        .antMatchers("/management/**").hasAuthority(AuthoritiesConstants.ADMIN)
        .antMatchers("/swagger-resources/configuration/ui").permitAll()
    .and()
        .apply(securityConfigurerAdapter());
}
\end{lstlisting}
Cet extrait de code désactive des options qui pourrait débouché sur des vulnérabilités si elles n'étaient pas bien utilisées\cite{noauthor_cross-site_nodate},
active une système de cession sans état\fnmark{}, ce qui est possible grâce à l'utilisation d'une authentification par \gls{a-jwt}
\fntext{Sans état signifie que le \gls{g-server} ne doit pas garder les connections de l'utilisateur en mémoire.}
et enfin configure les règles d'accès selon l'URL appelée.

Seul les utilisateurs authentifiés ont accès aux ressources derrière le chemin \lstinline{/api/**}.
Cela concerne par exemple le chemin \lstinline{/api/registration/validate} utilisé dans l'extrait de code de la sous-section \ref{subsubsec:spring-web}.

Les ressources réservées à l'administrateur sont protégées par le chemin \lstinline{/management/**} à l'exception du chemin \lstinline{/management/health} qui indique si le \gls{g-server} est en bonne santé ou non.

\subsubsection{L'injection de dépendance}
\label{subsubsec:dependency-injection}

\paragraph{}
Dans un programme, une partie d'un logiciel dépend très souvent d'autres parties du logiciel.
Si le bout de code considéré devait lui-même récupérer les bouts de code dont il a besoin, il y serait fortement couplé.
C'est à dire qu'il doit connaître le comportement du code dont il dépend sans quoi il ne sait pas s'en servir.
Ce n'est pas forcément une mauvaise mais dans certains cas, ce n'est pas idéal.

Un exemple classique est la connexion à la base de données.
Établir la connexion à la base de données requière des informations comme son emplacement, le nom d'utilisateur et le mot de passe.
Si chaque bout de code qui doit accéder à la base de données devaient aussi connaître ces informations, cela donnerait un code assez complexe.

C'est la qu'intervient l'injection de dépendance.
Plutôt que de devoir instancier\fnmark{} sa dépendance, c'est le \gls{g-framework} qui s'en charge et nous le fournit.
\fntext{Créer un exemplaire concret d'une classe ou d'un modèle}
Dans notre cas, c'est \Gls{g-spring} qui s'en charge.

\paragraph{}
Voici deux applications de l'injection de dépendance:
Voici un premier
\begin{lstlisting}[language=Java]
public LoginUnavailableValidator(LearnerRepository learnerRepository) {
    this.learnerRepository = learnerRepository;
}
// TRUNCATED
learnerRepository.findAllByLoginIn(rowsByLogin.keySet()) // TRUNCATED
\end{lstlisting}
Et un deuxième
\begin{lstlisting}[language=Java]
public ExistingUserValidator(UserExtendedResourceApiClient userExtendedResourceApiClient) {
	this.userExtendedResourceApiClient = userExtendedResourceApiClient;
}
// TRUNCATED
Map<String, UserModel> existingEmail = this.userExtendedResourceApiClient.getUserByLoginMapUsingPOST(emails).getBody();
// TRUNCATED
\end{lstlisting}
Le premier extrait de code fait appel à une base de données tandis que le deuxième appelle un \gls{a-api} en passant par le réseau.
Le développeur ne voit pas la différence car c'est \Gls{g-spring} qui s'est chargé de fournir les dépendances nécessaires.

En l'occurrence, il peut être intéressant de noter que la méthode du deuxième extrait de code est générée par un outil appelé Swagger sur base du code de la ressource.
Je parle bien d'une ressource telle que celle montrée en exemple dans la sous-section \ref{subsubsec:spring-web}.
Une autre différence est que le premier extrait de code vient de Click-and-Run tandis que le second est un extrait d'une service d'Altissia tournant actuellement en production.
La transition entre les deux est donc triviale.

\paragraph{}
Ce mécanisme est très utile pour implémenter le patron de conception de la stratégie, comme vous le verrez dans la sous-section \ref{subsec:dependency-injection}.
