L'élaboration de ce cahier des charges s'est faite sur bases de plusieurs interviews et d'échanges de questions et réponses.
Le développement sera géré selon les principes agiles\footnotemark.
Le cahier des charges n'est donc pas défini d'une traite mais tout au long de la création de l'application.
\footnotetext{Dans le contexte du développement logiciel, l'agilité définit un cadre de travail\cite{noauthor_methode_2018}.
Un point central est le développement par itération dont chacune produit un délivrable critiquable par le client.
Ceci a pour but de limiter les efforts potentiellement perdus si le résultat ne correspond pas aux attentes.}

\subsection{Les interviews}
\label{subsec:interviews}

    \paragraph{}
    Une interview avec le directeur technique m'a permis d'établir comment la nouvelle application intéragira avec les applications existantes.

    L'application sera hébergée sur le réseau de travail de l'entreprise.
    Elle doit supporter certaines fonctionalités mais pas l'authentification utilisée par nos microservices en tant que telle.
    Afin de donner accès aux ressources protégées, Altissia créera une version personnalisée\footnotemark de l'application.
    \footnotetext{La personnalisation signifie la reprise du code source afin d'y ajouter les fonctionalités désirées.}

    Ainsi, l'application sera constitué de quatres sortes de composants:
    \begin{itemize}
        \item Une application web cliente Angular qui lui permet d'intéragir avec les utilisateurs.
        \item Une application web server qui permet de communiquer avec les ressources réseaux.
        Celle-ci sera personnalisée afin d'implémenter l'authentification d'Altissia.
        \item Une librairie d'automatisation qui gère l'exécution des tâches et leur journalisation.
        \item Des multiples modules qui chacun permettent de définir une ou plusieurs tâches.
    \end{itemize}

    \paragraph{}
    Renaud, un développeur backend expérimenté, m'a expliqué le défi que sera la libraire d'automatisation de tâches.
    Il faut supporter la définition d'une tâche, l'asynchronicité de son exécution, la possibilité d'exécutions répétées,
    l'utilisation de ressources partagées et la réutilisabilité des tâches.

    Je dois lui proposer une architecture qui permet de relever tous ces défis.

    \paragraph{}
    Sophie, une cheffe de projet qui s'occupe notamment de la publication du contenu des cours et tests de niveaux,
    a défini quel serait le périmètre de l'application.
    Le but est de ne pas implémenter des fonctionalités qui font déjà l'objet de développements prévus ou en cours.

    La fonctionalité qui obtient la priorité pour le moment est la validation du contenu des fichiers Excel des questions du test de niveau.
    Celle-ci est fort demandée et sa complexité est moyenne.
    Elle permettrait de se faire une idée plus précise de ce que l'application pourrait apporter.

    \paragraph{}
    Le directeur technique et moi-même sommes mis d'accord pour une réunion bi-hebdomadaire.
    Je dois organisé une nouvelle réunion avec Renaud pour lui présenter mon plan d'architecture.
    Je verra Sophie dans un moi ou deux pour passer en revue le premier délivrable et discuter des prochains délivrables.

\subsection{Les demandes clients}
\label{subsec:customer-requests}

    La première fonctionalité sera la validation des fichiers Excel des questions de test de niveaux.
    Cette validation comprends les points suivants:
    \begin{enumerate}
        \item La correspondance à une expression régulière.
        \item L'unicité des valeurs au sein d'une colonne.
        \item La cohérence entre deux colonnes (Si la colonne type a la valeur audio, alors la colonne son doit avoir une valeur).
        \item La présence d'une colonne.
        \item L'absence d'impact de colonnes superflues.
        \item L'appartenance à un ensemble de comparaison.
    \end{enumerate}

\subsection{Les propositions au client}
\label{subsec:proposals-to-customer}

    J'ai proposé de n'implémenter que la présence d'une colonne pour la première itération.
    C'est évidemment trop peu pour réaliser un scénario complet mais l'objectif est la mise en place des différents composants logiciels.
    Ainsi, une fois cette première itération accomplie, on pourra juger de la pertinence de la solution choisie.
