\paragraph{}
Altissia est une entreprise travaillant dans le domaine de l'apprentissage des langues étrangères.
Elle propose des formations en lignes.
Elle développe les plateformes informatiques, crée le contenu de cours, fournit du support aux utilisateurs et héberge une
partie de son infrastructure.
Son catalogue contient 31 langues d'interfaces et 26 langues d'apprentissages\footnote{Information vraie au 11 Janvier 2019.}. % TODO Assert this information, see https://github.com/click-and-run/overleaf-thesis/issues/2
Elle se distingue de ses concurrents par son identité académique en proposant des cours complets permettant de passer de débutant à bilingue.

\paragraph{}
Son siège principal se trouve en Belgique mais elle a aussi des sièges en France, au Brésil, au Canada et au Maroc. % TODO Assert this information, see https://github.com/click-and-run/overleaf-thesis/issues/2

Environ 80 employés travaille en son sein dont 60 sur le site de Louvain-la-Neuve. % TODO Assert this information, see https://github.com/click-and-run/overleaf-thesis/issues/2

Nous formons un consortium avec le CLL et l'UCL afin de soutenir l'apprentissage des langues pour les participants du programme Erasmus+\cite{cll_actualites_nodate}{}.

Nos clients sont principalement des institutions comme l'Union Européenne ou la Région Wallonne, des écoles ou encore
des entreprises privées qui veulent former, respectivement, leurs citoyens, leurs étudiants et leurs collaborateurs.

\paragraph{}
Altissia propose des cours sur sites Internet et applications mobiles, des classes à distance, des articles de presses, un réseau social et des tables de conversations.
Et fait donc face à de nombreux concurrents présents sur différents segments de marché.
On peut citer Rosetta Stone, Babel, Duolinguo, MemRise, italki ou encore HelloTalk mais la liste est sans fin. %TODO ask who are our competitors

\paragraph{}
L'application sera utilisée par le service linguistique et informatique qui sont conjointement responsables du contenu de cours
et du développement des plate-formes de cours.
Ces deux services comptabilisent une quarantaine de personnes et comporte des profils de linguistes, développeurs informatiques,
concepteurs d'interfaces utilisateur, administrateurs systèmes, chefs de projets et analyste fonctionnels.
