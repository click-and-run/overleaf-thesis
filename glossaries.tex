\newglossaryentry{g-null}{
    name=null,
    description={est la valeur que l'on attribue à une variable pour dénoter l'absence de valeur, ce qui est différent d'une valeur de zéro pour un nombre par exemple}
}
\newglossaryentry{g-edi}{
    name=environnement de développement intégré,
    description={est un logiciel qui intègre les outils pour écrire le code source ainsi que les outils pour le transformer en application prête pour l'utilisateur final}
}
\newglossaryentry{g-server}{
    name=serveur,
    description={est une entité qui fournit une ressource à une entité qui lui a demandé cette ressource}
}
\newglossaryentry{g-client}{
    name=client,
    description={est une entité qui reçoit une ressource d'une entité à laquelle elle lui a demandée cette ressource}
}
\newglossaryentry{g-jwt}{
    name=json web token,
    description={est un standard industriel permettant de représenter des prétentions entre deux partis de manière sécurisée}
}
\newglossaryentry{g-hash}{
    name=hash,
    description={est le résultat d'une fonction de hachage}
}
\newglossaryentry{g-hash-func}{
    name={fonction de hachage},
    description={est une fonction qui calcule une empreinte numérique à partir d'une entrée donnée. Le résultat de cette fonction, le hash, sera toujours le même pour la même entrée et jamais le même pour une entrée différente. Il est très difficile, voir impossible de trouver l'entrée sur base du résultat, la fonction est donc à sens unique. Afin de pouvoir standardiser de telles fonctions dans le contexte de la sécurité informatique, elles sont conçues pour utiliser un secret qui modifie le résultat de la fonction. De telle sorte qu'il n'est pas possible de reproduire le comportement d'une fonction son connaître sans secret, cela alors que la fonction elle-même est connue}
}
\newglossaryentry{g-json}{
    name={JavaScript Object Notation},
    description={est un format textuel utilisée par le langage de programmation \textit{JavaScript} pour représenter des données structurées}
}
