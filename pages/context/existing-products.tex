\paragraph{}
Lors de la recherche d'une solution existante, nous avons concentré notre attention sur les points suivants:
\begin{itemize}
    \item Exécuter des programmes arbitraires
    \item Permettre de définir les paramètres d'un programme au travers d'une interface graphique
    \item Tenir un journal des scripts exécutés
    \item Notifier un utilisateur de la fin de l'exécution d'une tâche
    \item Chercher toute entité (script, exécution, résultat, utilisateur, etc.)
    \item La courbe d'apprentissage de l'outil
\end{itemize}
Il existe de nombreux outils sur le marché mais malheureusement, ils sont tous destinés à des développeurs...
Les plus connus sont Ansible, Rundeck, Salt, Chef et Fabric.

\subsection{Ansible}

\subsection{Rundeck}
Nous avons identifié Rundeck comme étant l'offre la plus proche de nos besoins.
Toutefois, l'outil cible plutôt des techniciens et reste relativement complexe à utiliser.
C'est d'ailleurs un outil que nous nos développeurs utilisent déjà, principalement pour des tâches de gestion des versions du code source.

\paragraph{}
Dans notre instance de Rundeck, il existe une tâche appelée \textit{Republication}.
Lancer cette tâche devrait être aussi simple que d'appuyer sur un bouton mais ce n'est malheureusement pas le cas.
Il faut cliquer quatre fois pour l'exécuter et il n'y a pas de champs de recherche.

De plus cette tâche peut être paramétrée, il est possible de choisir une langue à publier.
Sauf que pour ce faire, il faut définir une variable d'environnement\footnote{variable d'environnement: variable définie au niveau du système d'exploitation permettant de partager sa valeur à différentes applications. Ici, elle permet à Rundeck, et donc l'utilisateur, de communiquer avec l'application \textit{Republication}}.
Cette genre de technicité devrait être masqué afin de mettre l'outil dans les mains de n'importe qui.

% todo insérer la figure
