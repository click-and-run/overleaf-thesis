\paragraph{}
Différents produits auraient pu nous intéresser comme par exemple
\href{https://ifttt.com/}{IFTTT},
\href{https://zapier.com/}{Zapier} ou
\href{https://flow.microsoft.com/fr-fr/}{Microsoft Flow}.
Ces outils gère tous l'aspect de journalisation, leurs interfaces sont conviviales et elles remplissent certains de nos besoins métiers.
Quoiqu'il arrive, certains de nos besoins métiers sont uniques et nous devrons les implémenter nous même à travers des modules.
On pourrait par exemple tirer parti des \href{https://powerapps.microsoft.com/fr-fr/}{PowerApps} de \href{https://flow.microsoft.com/fr-fr/}{Microsoft Flow}.

\paragraph{}
Toutefois, dès lors que l'on sort de l'écosystème de l'outil que nous aurions choisis, le framework de ce dernier ne nous apporterait que peu d'aide.
Beaucoup de nos besoins métiers ne correspondent pas exactement à ces outils et nous serions laissés à nous mêmes.
Aussi, un problème majeur est le système d'authentification de nos différentes ressources internes;
Un accès \href{https://fr.wikipedia.org/wiki/Samba_(informatique)}{SAMBA} ou
une authentification sur un \href{https://fr.wikipedia.org/wiki/Annuaire}{annuaire} ne poserait pas de soucis
mais l'intégration avec nos micro services serait très complexe à mettre en oeuvre.

\paragraph{}
Ainsi, en développant un outil qui corresponde exactement à nos besoins et sur lequel nous avons totalement la main, nous espérons obtenir un meilleur résultat.
