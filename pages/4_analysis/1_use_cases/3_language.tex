\subsection{L'internalisation}
\label{subsec:i18n}

\paragraph{}
Chaque valeur textuelle statique\fnmark{} doit être traduite. Seuls des employés d'Altissia auront accès à cette application, il n'est donc pas nécessaire de traduire l'application dans toutes les langues supportées par les autres applications d'Altissia\fnmark{}.
\fntext{Toutes données qui ne viennent pas d'une base de données sont dites statiques, car elles ne peuvent pas changer pendant la vie de l'application.}
\fntext{Ce qui aurait fait une trentaine de langues dont je n'en connais que trois...}

Il est par contre obligatoire d'être compatible avec le système de localisation qui est déjà mis en place.

Altissia embarque les mécanismes nécessaires à la localisation dans chaque application. Une application peut donc changer de langue sans contacter de \gls{g-server}.

De plus, les informations de localisations sont stockées sur l'application web PhraseApp.
Les localisations devront donc être téléchargées depuis cette dernière.
Le format privilégié pour les applications clientes est le format \gls{a-json}, que l'application devra donc utiliser.
