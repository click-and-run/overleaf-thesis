\section{Lots d'informations identifiés}
\label{sec:identified-information-packages}
    Intégrer une nouvelle application dans un écosystème de logiciels existants imposent de conserver certaines informations, d'en remplacer certaines et d'en créer de nouvelles.

\subsection{Lots d’informations existants à conserver}
\label{ssec:existing-information-packages-to-keep}
    L'application va essentiellement en remplacer une autre, il est donc évident que des informations doivent être conservées.
    Le logiciel doit utiliser:
    \begin{itemize}
        \item Les enregistrements des utilisateurs pour pouvoir les identifier; ils présents dans la base de données du système d'authentification.
        \item Les licences pour contrôler les droits des utilisateurs; elles sont stockées dans la base des données du service de licences.
        \item Les fichiers Excel contenant les cours, la matière première de l'application\fnmark.
        \item Les règles de validation implémentées dans Altissia launcher\fnmark.
    \end{itemize}
    
    % TODO Demander l'autorisation
    \fntext{Ces données sont propriétaires et je ne peux en partager qu'un cours extrait, voir l'annexe \ref{ch:leveltest-sample}}
    \fntext{Pour un court extrait, voir l'annexe \ref{ch:altissia-launcher-code}}

\subsection{Lots d’informations existants à remplacer}
\label{ssec:existing-information-packages-to-replace}

    Les données déployées sur les serveurs doivent être remplacées à chaque mise à jour de contenu.
    Cette étape est cruciale car tant qu'elle n'est pas achevée, tout le travail de l'équipe linguistique est indisponible aux utilisateurs.

\subsection{Lots d’informations à produire}
\label{ssec:information-packages-to-produce}

    \paragraph{}
    Le contenu de cours produit par les linguistes doit être consommé par l'application pour fournir le contenu des plateformes.
    
    \paragraph{}
    Des rapports doivent être fournis aux utilisateurs à chaque validation afin qu'ils sachent comment corriger le contenu.
    
    \paragraph{}
    L'historique des exécutions des scripts doit être construites lors de l'utilisation du nouvel outil.
    % TODO parler des sauvegardes lorsque la fonctionnalité est introduite.
    