\subsection{Les interviews}
\label{subsec:interviews}

\paragraph{}
J'ai eu une première interview avec Grégory pour définir le besoin général et le contexte qui l'entoure.
Il explique qu'en facilitant la vérification et la conversion des nouveaux contenus des cours, l'application permettrait d'accélérer l'intégration des nouveaux contenus de cours dans nos plateformes.

Nous discutons de la forme que cette application va prendre. 
Je propose de la structurer en trois parties: une librairie applicative, un serveur et une libraire de composants webs.
Après discussions, nous concluons qu'il n'est pas utile de former une librairie et que de directement intégrer les fonctionnalités dans un serveur simplifiera le développement.

Il s'agit donc de créer une application serveur qui s'occupe de la manipulation des données et une application cliente qui affiche les données et enregistre les choix de l'utilisateur.
Sous cette forme, Altissia pourra facilement personnaliser \fnmark l'application pour l'intégrer aux logiciels existants.
\fntext{Il s'agit ici de faire un \textit{fork} de l'application. Une sorte de copie que l'on personnalise sans modifier l'originale, mais qui peut toujours obtenir les dernières mises à jour de la source.}

\paragraph{}
Ensuite, j'ai eu une réunion avec Renaud et Sophie.
À deux, ils ont défini précisément les problèmes de l'outil existant et pourquoi on veut le remplacer.

Renaud connait très bien l'outil que l'on veut remplacer, Altissia launcher, il m'a expliqué les difficultés rencontrées pendant son développement et comment il a terminé dans son état actuel.
La principale cause est le temps de développement nécessaire pour prendre en charge de nouveaux formats et la difficulté d'utilisation qui encourage les utilisateurs à s'en passer.

Sophie a établi la liste des manquements du logiciel actuel qu'il faudrait suppléer et les fonctionnalités qu'elle voudrait ajouter.
Certaines règles ne sont pas vérifiées, les codes d'erreur sont cryptiques et certaines informations pourraient accompagner les erreurs afin d'en faciliter la correction.

\paragraph{}
Par la suite, nous avons continué de nous rencontrer dans le cadre des réunions SCRUM:
\begin{itemize}
    \item Le \textit{daily standup}: courte réunion quotidienne pour informer de l'avancement et toute difficulté rencontrée
    \item Le \textit{sprint planning}: Réunion bimensuelle où l'on planifie les deux prochaines semaines
    \item Le \textit{backlog refinement}: Réunion bimensuelle où l'on clarifie la définition des fonctionnalités pour s'assurer qu'elles sont correctement comprises et prêtes à être implémentées.
    \item La \textit{Review}: Réunion bimensuelle en fin de sprint où l'on présente les résultats produits durant la période à toutes personnes intéressées.
\end{itemize}
